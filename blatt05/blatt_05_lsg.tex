\documentclass[a4paper, 11pt]{article}
\usepackage{geometry}
\usepackage{graphicx}
\usepackage{a4wide}
\usepackage{ulem}
\usepackage{amsthm}
\usepackage{amsmath}
\usepackage{amsfonts}
\usepackage{amssymb}
\usepackage[T1]{fontenc}
\usepackage{ngerman}
\usepackage{graphicx}
\usepackage{epic}
\usepackage{enumerate}
\usepackage{tabu}
\usepackage [latin1]{inputenc}
\geometry{a4paper,left=25mm,right=25mm,top=10mm,bottom=15mm}
%\renewcommand{\baselinestretch}{1.5}
\newcommand{\ol}{\overline}
\newcommand{\makeline}{\hrule\vspace{5pt}}
\newcommand{\ip}[2]{\left< #1, #2 \right>}

\title{5. �bungsblatt zu Software Qualit�t}
\author{Michel Meyer, Manuel Schwarz}

\begin{document}
  \maketitle

  \section*{Aufgabe 5.1}
  \subsection*{(a)}
  \textit{Hier das Bild rein...}

  \subsection*{(b)}
  \begin{tabular}[c]{|l|l|l|}\hline
  \textbf{Kategorie} & \textbf{ID} & \textbf{Pfad} \\\hline
  Ohne Schleife & A0 & $n_s, n_1, n_f$ \\\hline
                & B0 & $n_s, n_1, n_2, n_5, n_f$ \\\hline

  \end{tabular}


\end{document}
