\documentclass[a4paper, 11pt]{article}
\usepackage{geometry}
\usepackage{graphicx}
\usepackage{a4wide}
\usepackage{ulem}
\usepackage{amsthm}
\usepackage{amsmath}
\usepackage{amsfonts}
\usepackage{amssymb}
\usepackage[T1]{fontenc}
\usepackage{german}
\usepackage{graphicx}
\usepackage{epic}
\usepackage{enumerate}
\usepackage [latin1]{inputenc}
\geometry{a4paper,left=25mm,right=25mm,top=10mm,bottom=15mm}
\renewcommand{\baselinestretch}{1.5}
\newcommand{\ol}{\overline}
\newcommand{\makeline}{\hrule\vspace{5pt}}
\newcommand{\ip}[2]{\left< #1, #2 \right>}

\title{1. �bungsblatt zu Software Qualit"at}
\author{Michel Meyer, Manuel Schwarz}

\begin{document}
  \maketitle

  \section*{Aufgabe 1.1}
	\subsection*{(a)}
	\begin{itemize}
		\item \textbf{Portabilit�t}: Plattformunabh�ngigkeit durch JAVA gegeben.
		\item \textbf{Vollst�ndigkeit}: W�hrend der Programmierung wurde sichergestellt, dass alle in der Aufgabe genannten Punkte implementiert wurden.
		\item \textbf{Benutzbarkeit}: Intuitive, traditionelle und schlichte GUI, die die Benutzung schnell und einfach h�lt.
	\end{itemize}
	\subsection*{(b)}
	\begin{itemize}
		\item \textbf{Sicherheit}: Es k�nnten Dialog-Abfragen implementiert werden, sodass nicht jede beliebige Datei (versehentlich) �berschrieben werden kann.
		\item \textbf{Wartbarkeit}: Javadoc erh�ht die Wartbarkeit sowohl f�r den/die urspr�nglichen Programmierer als auch f�r weitere Leute, die sich in den Code einarbeiten m�ssen.
		\item \textbf{Benutzbarkeit/Robustheit}: Hinsichtlich eines Text-Editors sollten verschiedene Text-Kodierungen unterst�tzt werden.
		\item \textbf{Effizienz}: Steigerung der Effizienz durch Optimierung der Methoden einzelner Aktionen (laden, speichern, usw.).
	\end{itemize}
	\subsection*{(c)}
	\begin{itemize}
		\item \textbf{Syntaxfehler:} Das Programm lie� sich nicht kompilieren, weil die JAVA-Syntax nicht eingehalten wurde.
		\item \textbf{Semantischer Fehler}: Diverse \texttt{NullPointerException}s, wenn beim \texttt{JFileChooser} auf "`Abbrechen"' geklickt wurde, wodurch \texttt{null} an verschiedene Methoden weitergeleitet wurde.
		\item \textbf{Semantischer Fehler}: Es wurde beim Beenden/Laden/Neu nicht immer darauf hingewiesen, dass das aktuelle Dokument ge�ndert wurde, weil die Listener nicht korrekt konfiguriert wurden.
		\item \textbf{Optimierungsfehler}: Beim Versuch eine Methode k�rzer und schneller (genauer: if-Bedingungen zusammenzufassen) zu machen, machte die Methode in den meisten F�llen gar nichts mehr.
	\end{itemize}
	\subsection*{(d)}
	Die Fehlerklassifikation dient zum einen einer Priorit�tseinteilung. So sind beispielsweise Syntaxfehler eher zu beheben als semantische, weil bei fehlerhafter Syntax in der Regel gar nichts funktioniert (Beispiel in JAVA: Der Code l�sst sich nicht kompilieren).\\
	Zum anderen dient die Klassifikation der Gruppenzuweisung. So k�nnen Optimierungsfehler von Teams behoben werden, die sich in der Optimierung auskennen, w�hrend Spezifikationsfehler oder Parallelit�tsfehler an entsprechende andere Teams weitergeleitet werden k�nnen.
  Des Weiteren k�nnen die QS-Prozesse bei folgenden Produkten verbessert werden, da man mehr Kontrolle an h"aufigen Fehlerquellen erwirken kann.
	\subsection*{(e)}
	Mit Hilfe der expliziten Qualit�tssicherungsma�nahmen werden die vom Programmierer gemachten impliziten
  Ma�nahmen sichergestellt, unterst�tzt und eventuell besser umgesetzt. Zum Beispiel k�nnten die expliziten
  Ma�nahmen zur erh�hten Sicherheit (Einbauen von Dialog-Abfragen) die impliziten �berlegungen und Umsetzungen
  zur Benutzbarkeit verbessern (kein versehentliches �berschrieben von Dateien).\\
  Des Weiteren k�nnen durch implizite, die Programmiersprache betreffende Ma�nahmen bereits Vorteile entstehen
  (z.B. Plattformunabh�ngigkeit bei JAVA oder ordentlich formatierter Code bei Python).
  Explizite Ma�nahmen kosten zu Beginn eventuell mehr Zeit, zahlen sich am Ende jedoch oft aus
  (bessere Wartbarkeit durch guten Code / gute Dokumentation, besseres Endprodukt mit h�herer Qualit�t).
  \section{Aufgabe 1.2}
  \begin{enumerate}[(a)]
    \item \begin{enumerate}[1.]
            \item \textbf{Qualit"atsma\ss :} meantime between failures (MTFB)
            \item \textbf{Qualit"atsmerkmal}: Zuverl"assigkeit
            \item \textbf{Auspr"agung:} $>$ 500s
          \end{enumerate}
    \item \begin{enumerate}[1.]
            \item \textbf{Qualit"atsmerkmal:} Benutzbarkeit (Usability)
            \item \textbf{Qualit"atsteilmerkmal:} Verst"andlichkeit
            \item \textbf{Qualit"atsma\ss :} Befragung von Testnutzern
            \item \textbf{Auspr"agung:} Skala von 1 - 10 (wobei 1 = unverst"andlich und 10 = intuitiv)
          \end{enumerate}
    \item $\textbf{Verf"ugbarkeit} = \frac{MTBF}{MTBF + MTTR}$\\
          mit $MTBF = \frac{Betriebsdauer}{Fehleranzahl}$\\
          $MTBF = \frac{(t2 - t1) + (t4 - t3) + (t6 - t5)}{3}$\\
          $MTBF = \frac{23}{3} = 7.67$\\
          daraus folgt:\\
          $Verf"ugbarkeit = \frac{7.67}{7.67 + ((t3 - t2) + (t5 - t4))}$\\
          $Verf"ugbarkeit = \frac{7.67}{12.67} = 0.605$
  \end{enumerate}
\end{document}
