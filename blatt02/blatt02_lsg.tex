\documentclass[a4paper, 11pt]{article}
\usepackage{geometry}
\usepackage{graphicx}
\usepackage{a4wide}
\usepackage{ulem}
\usepackage{amsthm}
\usepackage{amsmath}
\usepackage{amsfonts}
\usepackage{amssymb}
\usepackage[T1]{fontenc}
\usepackage{german}
\usepackage{graphicx}
\usepackage{epic}
\usepackage{enumerate}
\usepackage [latin1]{inputenc}
\geometry{a4paper,left=25mm,right=25mm,top=10mm,bottom=15mm}
\renewcommand{\baselinestretch}{1.5}
\newcommand{\ol}{\overline}
\newcommand{\makeline}{\hrule\vspace{5pt}}
\newcommand{\ip}[2]{\left< #1, #2 \right>}

\title{2. �bungsblatt zu Software Qualit�t}
\author{Michel Meyer, Manuel Schwarz}

\begin{document}
  \maketitle

  \section*{Aufgabe 2.1}
	\subsection*{(a)}
	\subsection*{(b)}
	\section*{Aufgabe 2.2}
  \subsection*{(a)}
  Die folgende Tabelle zeigt die g�ltigen �quivalenzklassen (G�) und die ung�ltigen �quivalenzklassen (U�):
  \begin{center}
		\begin{tabular}[h]{|l|l|p{4cm}|p{4cm}|}\hline
			-               & Vorgabe                 & G�                                                                  & U�                                                                  \\\hline
			Material        & Ton, Marmor, Granit     & \parbox{4cm}{$x$ = "`Ton"' \\ $x$ = "`Marmor"' \\ $x$ = "`Granit"'} & \parbox{4cm}{$x$ ist nicht "`Ton"',\\ "`Marmor"' oder "`Granit"'}   \\\hline
			L�nge           & $17cm \leq x \leq 68cm$ &    &    \\\hline
			Menge           &         &    &    \\\hline
			Auftragsnummer  &         &    &    \\\hline
		\end{tabular}
	\end{center}
	Siehe da
  \subsection*{(b)}
  \subsection*{(c)}
\end{document}
