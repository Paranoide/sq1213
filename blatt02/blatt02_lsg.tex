\documentclass[a4paper, 11pt]{article}
\usepackage{geometry}
\usepackage{graphicx}
\usepackage{a4wide}
\usepackage{ulem}
\usepackage{amsthm}
\usepackage{amsmath}
\usepackage{amsfonts}
\usepackage{amssymb}
\usepackage[T1]{fontenc}
\usepackage{german}
\usepackage{graphicx}
\usepackage{epic}
\usepackage{enumerate}
\usepackage{tabu}
\usepackage [latin1]{inputenc}
\geometry{a4paper,left=25mm,right=25mm,top=10mm,bottom=15mm}
\renewcommand{\baselinestretch}{1.5}
\newcommand{\ol}{\overline}
\newcommand{\makeline}{\hrule\vspace{5pt}}
\newcommand{\ip}[2]{\left< #1, #2 \right>}

\title{2. �bungsblatt zu Software Qualit�t}
\author{Michel Meyer, Manuel Schwarz}

\begin{document}
  \maketitle

  \section*{Aufgabe 2.1}
	\subsection*{(a)}
	Das magische Dreieck veranschaulicht die Beziehung zwischen den Komponenten \textbf{Qualit�t}, \textbf{Kosten} und \textbf{Zeit} eines (Software-)Projektes. Soll in eine dieser Komponenten optimiert werden (d.h.\ h�here Qualit�t, weniger Kosten oder weniger Zeit), so muss mindestens eine der beiden anderen Komponenten vernachl�ssigt werden.\\
	Veranschaulicht wird diese Beziehung meist durch drei orthogonale Achsen, die jeweils f�r eine der drei Komponenten stehen und auf denen ein Punkt die "`H�he"' der Komponente symbolisiert. Diese Punkte k�nnen nun verschieden angepasst werden, allerdings muss die dreieckige Fl�che, die die drei Punkte aufspannen, immer gleich bleiben (die Fl�che steht f�r die verf�gbaren Gesamtressourcen).
	\subsection*{(b)}
	Beispiele:
	\begin{itemize}
		\item M�chte man in der gleichen Zeit eine \textbf{h�here Qulit�t} erreichen, so muss man mehr Kosten in Anspruch nehmen.
		\item M�chte man bei gleichen Kosten die \textbf{Zeit verk�rzen}, so entstehen Abz�ge in der Qualit�t.
		\item Etwas widerspr�chlich wird es im folgenden Beispiel: Man l�sst die Qualit�t konstant, \textbf{optimiert die Kosten} und nimmt daf�r eine l�ngere Zeit in Anspruch. Jedoch fallen bei l�nger dauernden Projekten auch immer h�here Laufkosten an.
	\end{itemize}
	\section*{Aufgabe 2.2}
  \subsection*{(a)}
  Die folgende Tabelle zeigt die g�ltigen �quivalenzklassen (G�) und die ung�ltigen �quivalenzklassen (U�):
  \begin{center}
		\begin{tabular}[h]{|l|p{4,8cm}|p{3,5cm}|p{4,5cm}|}\hline
			-               & Vorgabe                                                    & G�                                                                  & U�                                                                  \\\hline
			Material        & \parbox{5cm}{$x \in$\\ \{"`Ton"', "`Marmor"', "`Granit"'\}}& \parbox{4cm}{$A_1$ = "`Ton"' \\ $A_2$ = "`Marmor"' \\ $A_3$ = "`Granit"'} & \parbox{5cm}{$A_7$ ist nicht "`Ton"',\\ "`Marmor"' oder "`Granit"'}   \\\hline
			L�nge           & $17cm \leq x \leq 68cm$                                    & $17cm \leq A_4 \leq 68cm$                                             & \parbox{5cm}{$A_8 < 17cm$ \\ $A_9 > 68cm$}                              \\\hline
			Menge           & $1 \leq x \leq 9999$                                       & $1 \leq A_5 \leq 9999$                                                & \parbox{5cm}{$A_{10} < 1$ \\ $A_{11} > 9999$}                                 \\\hline
			Auftragsnummer  & \parbox{4cm}{Beginnt mit "`F"',\\endet mit "`2"'}          & \parbox{4cm}{$A_6$ beginnt mit "`F"'\\ und endet mit "`2"'}                   & \parbox{5cm}{$A_{12}$ beginnt nicht mit "`F"',\\ $A_{13}$ endet nicht mit "`2"'} \\\hline
		\end{tabular}
	\end{center}

  \subsection*{(b)}
  \footnotesize
  \begin{center}
    \begin{tabu}{|l||c|c|c|[1pt]c|c|c|c|c|c|c|}\hline
      Testf�lle           & 1               & 2   & 3   & 4   & 5   & 6   & 7   & 8   & 9   & 10  \\\hline\hline
      getestete �-Klassen & A1, A4, A5, A6  & A2  & A3  & A7  & A8  & A9  & A10 & A11 & A12 & A13 \\\hline
      Fliesenart          & Ton   & Marmor & Granit    & Stein & Ton & Marmor & Granit & Ton & Marmor & Granit \\\hline
      Kantenl�nge         & 17              & 68  & 17  & 68  & 16  & 69  & 17  & 68  & 17  & 68  \\\hline
      Liefermenge         & 1             & 9999  & 1 & 9999  & 1 & 9999  & 0 & 10000 & 1 & 9999  \\\hline
      Auftragsnummer      & F12             & F22 & F32 & F42 & F52 & F62 & F72 & F82 & A92 & F93 \\\hline
    \end{tabu}
  \end{center}
  \normalsize
  Eine obere Grenze f�r die Kantenl�nge liegt jeweils bei den Testf�llen 2, 4, 5, 8 und 10 vor.\\
  Eine untere Grenze f�r die Kantenl�nge liegt dementsprechend bei den Testf�llen 1, 3, 6, 7 und 9 vor.\\
  Die obere Grenze f�r die Liefermenge liegt bei den Testf�llen 2, 4, 6, 7 und 10 vor.\\
  Schlie�lich liegt die untere Grenze f�r die Liefermenge bei den Testf�llen 1, 3, 5, 8 und 9 vor.

  \subsection*{(c)}
\end{document}
