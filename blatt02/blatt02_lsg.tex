\documentclass[a4paper, 11pt]{article}
\usepackage{geometry}
\usepackage{graphicx}
\usepackage{a4wide}
\usepackage{ulem}
\usepackage{amsthm}
\usepackage{amsmath}
\usepackage{amsfonts}
\usepackage{amssymb}
\usepackage[T1]{fontenc}
\usepackage{german}
\usepackage{graphicx}
\usepackage{epic}
\usepackage{enumerate}
\usepackage [latin1]{inputenc}
\geometry{a4paper,left=25mm,right=25mm,top=10mm,bottom=15mm}
\renewcommand{\baselinestretch}{1.5}
\newcommand{\ol}{\overline}
\newcommand{\makeline}{\hrule\vspace{5pt}}
\newcommand{\ip}[2]{\left< #1, #2 \right>}

\title{2. �bungsblatt zu Software Qualit�t}
\author{Michel Meyer, Manuel Schwarz}

\begin{document}
  \maketitle

  \section*{Aufgabe 2.1}
	\subsection*{(a)}
	\subsection*{(b)}
	\section*{Aufgabe 2.2}
  \subsection*{(a)}
  Die folgende Tabelle zeigt die g�ltigen �quivalenzklassen (G�) und die ung�ltigen �quivalenzklassen (U�):
  \begin{center}
		\begin{tabular}[h]{|l|p{4,8cm}|p{3,5cm}|p{4,5cm}|}\hline
			-               & Vorgabe                                                    & G�                                                                  & U�                                                                  \\\hline
			Material        & \parbox{5cm}{$x \in$\\ \{"`Ton"', "`Marmor"', "`Granit"'\}}& \parbox{4cm}{$x_1$ = "`Ton"' \\ $x_2$ = "`Marmor"' \\ $x_3$ = "`Granit"'} & \parbox{5cm}{$x_7$ ist nicht "`Ton"',\\ "`Marmor"' oder "`Granit"'}   \\\hline
			L�nge           & $17cm \leq x \leq 68cm$                                    & $17cm \leq x_4 \leq 68cm$                                             & \parbox{5cm}{$x_8 < 17cm$ \\ $x_9 > 68cm$}                              \\\hline
			Menge           & $1 \leq x \leq 9999$                                       & $1 \leq x_5 \leq 9999$                                                & \parbox{5cm}{$x_{10} < 1$ \\ $x_{11} > 9999$}                                 \\\hline
			Auftragsnummer  & \parbox{4cm}{Beginnt mit "`F"',\\endet mit "`2"'}          & \parbox{4cm}{$x_6$ beginnt mit "`F"'\\ und endet mit "`2"'}                   & \parbox{5cm}{$x_{12}$ beginnt nicht mit "`F"',\\ $x_{13}$ endet nicht mit "`2"'} \\\hline
		\end{tabular}
	\end{center}
	Siehe da
  \subsection*{(b)}
  \subsection*{(c)}
\end{document}
