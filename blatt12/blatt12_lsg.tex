\documentclass[a4paper, 11pt]{article}
\usepackage{geometry}
\usepackage{graphicx}
\usepackage{a4wide}
\usepackage{ulem}
\usepackage{amsthm}
\usepackage{amsmath}
\usepackage{amsfonts}
\usepackage{amssymb}
\usepackage[T1]{fontenc}
\usepackage{ngerman}
\usepackage{graphicx}
\usepackage{epic}
\usepackage{enumerate}
\usepackage{tabu}
\usepackage [latin1]{inputenc}
\usepackage{algorithmic}
\usepackage{algorithm}
\usepackage{listings}
\usepackage{color}
\geometry{a4paper,left=15mm,right=25mm,top=20mm,bottom=25mm}
%\renewcommand{\baselinestretch}{1.5}
\newcommand{\ol}{\overline}
\newcommand{\makeline}{\hrule\vspace{5pt}}
\newcommand{\ip}[2]{\left< #1, #2 \right>}

\title{12. �bungsblatt zu Software Qualit�t}
\author{Michel Meyer, Manuel Schwarz}

\begin{document}
  \maketitle

  \section*{Aufgabe 12.1 - Anomalieanalyse}
  \subsection*{(a) Java-Beispiele}
  \subsubsection*{Schnittstellenanomalie}
    \lstinputlisting[caption={Schnittstellenanomalie} \label{lst:schnittstellenanomalie}, captionpos=t,language=JAVA]
      {Schnittstellenanomalie.java}
  Die Schnittstellenanomalie wird vom Compiler erkannt und erzeugt einen Fehler.
  \vspace{10mm}

  \subsubsection*{Variablendeklaration-/-nutzungsanomalie}
    \lstinputlisting[caption={Variablendeklarations-/-nutzungsanomalie} \label{lst:variablendeklarationsanomalie}, captionpos=t,language=JAVA]
      {Variablenanomalie.java}
  Die Variablendeklarations-/-nutzungsanomalie f�hrt ebenfall zu einem Fehler, der durch den Compiler erkannt wird.
  \vspace{10mm}

  \subsubsection*{Kontrollflussanomalie}
    \lstinputlisting[caption={Kontrollflussanomalie} \label{lst:kontrollflussanomalie}, captionpos=t,language=JAVA]
      {Kontrollflussanomalie.java}
  Die Kontrollflussanomalie f�hrt ebenfalls (in Java) zu einem Fehler, falls der Code nicht erreichbar ist.
  \vspace{10mm}

  \subsubsection*{Datenflussanomalie}
    \lstinputlisting[caption={Datenflussanomalie} \label{lst:datenflussanomalie}, captionpos=t,language=JAVA]
      {Datenflussanomalie.java}
  Die Datenflussanomalie kann (in Java) zu einem Fehler f�hren, wenn eine Variable beispielsweise nicht initialisiert wurde,
  bevor mit ihr gerechnet wird. Doppelzuweisungen f�hren dagegen zu keinem Fehler.

\end{document}
